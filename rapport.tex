\documentclass[a4paper]{article}
\usepackage[latin1]{inputenc} 
\usepackage[T1]{fontenc}
\usepackage[francais]{babel}

% Marc: ceci est un fichier latex.
% Tu peut rajouter ce que tu veux entre les sections
% Et Koliaza ou moi transformeront le fichier en pdf.
% Si tu veux �crire aussi en latex, il te faudra apprendre sur internet plus en d�tail
% Saches aussi que installer le package pour compiler le latex prend plus d'un Giga d'installation
\author{Nicolas Blanchard, Axel Davy et Marc Heinrich}
\title{Pr�sentation de notre simulateur}
\begin{document}
\maketitle
\section{Comment ex�cuter notre compilateur}

\section{Options disponibles}

\section{Difficult�s rencontr�es}
Apr�s la premi�re s�ance du TD, nous avions une fonction scheduler qui permettait de faire un tri topologique et de mettre ainsi dans l'ordre les instructions d'un fichier netlist source. Mais nous avons eu beaucoup de mal � avoir une fonction qui marche aussi pour les registres. En effet il faut que la fonction ne d�tecte pas de cycle combinatoire quand il y a une boucle caus�e par un registre. De plus si deux registres sont chacuns connect�s � la sortie de l'autre, on ne peut en ex�cuter un avant l'autre. Nous avons donc d�cid�s d'enlever les liaisons des registres dans le graphe du tri topologique, ainsi le tri se fait sans consid�rer les registres, se qui coupe les �ventuels cycles combinatoire qui auraient �t� r�p�r�s � cause d'une boucle de registre. Nous avons aussi fait en sorte que les registres donnent leurs sorties au d�but de l'�x�cution et capturent leur entr�es � la fin de l'�x�cution. Pour cela nous avons mis les registres dans un liste pour donner leur sortie au d�but de l'ex�cution et nous les avons mis � la fin de la liste � ex�cuter pour qu'ils puissent capturer leur entr�e.
Comme le type ``programme'' �tait incomplet � notre go�t on a cr�e un nouveau type Mprogramme : On souhaitait une liste des cas � traiter � part (comme les registres) et nous voulions que les �quations soient compos�es de cl�s indiquant la valeur de leur arguments dans un tableau.


\end{document}
